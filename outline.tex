\chapter*{Outline}
%
The document is organised as follows:
%
\begin{description}
\item{\bf{\ref{chap:intro}$~~~$Introduction}} 
\\ After a short review of recent achievements in particle physics, the motivation for measuring the top quark mass is given.

\item{\bf{\ref{chap:topphysics}$~~~$Top quark physics}} 
\\ After a short introduction to the \gls{SM}, the basic concepts of top quark physics are explained, followed by an overview of recent experimental results.

\item{\bf{\ref{chap:atlas}$~~~$The ATLAS experiment}} 
\\ An introduction to the \gls{ATLAS} detector and its main components is given.

\item{\bf{{\ref{chap:trackingupgrades}$~~~$Pixel detector upgrades for \RunTwo}}} 
\\ A report on the recent upgrades of the ATLAS pixel detector, including the \gls{IBL} detector, is given. This documents the results of one year hardware work at \gls{CERN}.

\item{\bf{\ref{chap:topmass7TeV}$~~~$Measurement of the \tquark\ mass at \boldmath$\sqrts=7$~\TeV}}
\\ A measurement of the top quark mass in the \dil\ channel using \gls{ATLAS} data at $\sqrts = 7$~\TeV\ is presented.

\item{\bf{\ref{chap:topmass8TeV}$~~~$Measurement of the \tquark\ mass at \boldmath$\sqrts=8$~\TeV}}
\\ A measurement of the top quark mass in the \dil\ channel using \gls{ATLAS} data at $\sqrts = 8$~\TeV\ is presented.

\item{\bf{\ref{chap:combinations}$~~~$Combinations of \tquark\ mass measurements}}
\\ Combinations of the \tquark\ mass measurements carried out in this work and the measurement in the \ljets\ channel at $\sqrts=7$~\TeV\ are performed.

\item{\bf{\ref{chap:unfolding}$~~~$An analysis using unfolded ATLAS data}}
\\ Using an unfolding method, the \tquark\ mass is measured at stable particle level.

\item{\bf{\ref{chap:nlo}$~~~$\Tquark\ mass analyses in the light of full \gls{NLO} calculations}}
\\ With the help of a new calculation of the process \ppWWbb\ at \gls{NLO}, the impact of theoretical uncertainties on the \tquark\ mass measurement is investigated.

\item{\bf{\ref{chap:conclusion}$~$Conclusion}}
\\ After a summary of the work, the prospects of \tquark\ physics during \RunTwo\ and the further operation of the \gls{LHC} are given.
\end{description}
\glsreset{ATLAS}
\glsreset{CERN}
\glsreset{NLO}

Natural units are used for physics quantities throughout this thesis, i.e. $c=\hbar=1$. Consequently, masses, momenta and energies carry the same unit. Acronyms are listed in a glossary at the end of this document.
\todo{remove `Contents" header}
\todo{no `between' in non-spatial meaning - replace with `of'}