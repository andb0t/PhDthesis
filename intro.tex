\chapter{Introduction}
\label{chap:intro}
%
Since the early days of particle physics in the beginning of the 20th century, the understanding of the fundamental principles of nature has grown rapidly. Not even a century ago, Ernest Rutherford proposed the existence of the proton as the basic constituent of the atomic nucleus. Since then, a vast number of new particles have been discovered, most of them composite, built up from a small set of elementary particles. Radical developments in mathematical physics and experimental techniques have paved the way to the fundamental theory of quantum fields, describing the particles and their interactions, a theory known today as the \gls{SM}. 
%
The pioneering theoretical work of Glashow, Salam and Weinberg, who were awarded the Nobel prize for physics in 1979, was soon followed by experimental discoveries, proving the validity of the \gls{SM} with extraordinary accuracy. Since then, the \gls{SM} has been tested in countless experiments and is today established as the basic theory of particle physics. However, the \gls{SM} shows a number of limitations and is therefore considered an effective theory, with the underlying fundamental theory still to be discovered. 


The heaviest of all known elementary particles is the \tquark, which thus plays a special role in the \gls{SM}. Its mass \mt\ is a fundamental parameter of \gls{QCD}, affecting cross section predictions with implications for \Hboson\ physics and the search for signs of physics \gls{BSM}. Furthermore, a precise knowledge of this parameter allows for consistency assessments of the \gls{SM} in electroweak fits, complementing direct searches for new physics phenomena. 
%
Modern accelerators give access to the energy regime of \tquark\ physics. The \gls{LHC} at \gls{CERN} enables abundant production of \tquark{s} and a precise assessment of their properties.
%
This work presents measurements of \mt\ in the \dileptonic\ \tquark\ pair decay channel and investigates experimental and theoretical aspects of its precise determination. 


The document is organised as follows:
%
\chap~\ref{chap:topphysics} gives a short introduction to the \gls{SM}, the basic concepts of top quark physics and an overview of experimental techniques to determine the \tquark\ mass.
%
The multipurpose detector \gls{ATLAS} and its main components are introduced in \chap~\ref{chap:atlas}.
%
Upgrades for the \gls{ATLAS} pixel detector are reported on in \chap~\ref{chap:trackingupgrades}, including the commissioning of the new \gls{IBL} detector. This includes the results of one year hardware work at \gls{CERN}.
%
A measurement of the \tquark\ mass in the \dil\ channel using \gls{ATLAS} data at $\sqrts = 7$~\TeV\ is presented in \chap~\ref{chap:topmass7TeV}.
%
Exploiting the larger data statistics collected at $\sqrts = 8$~\TeV, a multivariate analysis is used to obtain the most precise \tquark\ mass measurement in the \ttbarll\ decay channel to date and the first \tquark\ mass result at \gls{ATLAS} with a precision below $1$~\GeV. This is presented in \chap~\ref{chap:topmass8TeV}.
%
In \chap~\ref{chap:combinations}, combinations of the \gls{ATLAS} \tquark\ mass measurements in the \dil\ and \ljets\ channels are performed, further improving the achieved precision.
%
Using an unfolding method to correct the data for detector effects, the first steps towards a \tquark\ mass measurement at \stablevel\ are described in \chap~\ref{chap:unfolding}.
%
With the help of a new calculation of the process \ppWWbb\ including \gls{NLO} \gls{QCD} corrections, the impact of theoretical uncertainties on the \tquark\ mass measurement is investigated in \chap~\ref{chap:nlo}.
%
Finally, \chap~\ref{chap:conclusion} gives a summary of the work.


Natural units are used for physics quantities throughout this thesis, i.e. $c=\hbar=1$. Consequently, masses, momenta and energies carry the same unit, \GeV. Acronyms are listed in a glossary at the end of this document.
