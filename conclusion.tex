\chapter{Conclusions and outlook}
\label{chap:conclusion}
%
The \tquark\ mass is a \gls{SM} parameter with large impact on electroweak calculations, making it a decisive element for consistency assessments of the \gls{SM} and a vital ingredient for \gls{BSM} predictions. Therefore, its precise determination is of prime importance for the advancement of particle physics.
%
Precision measurements of the \tquark\ mass require optimal detector performance, advanced analysis techniques and a careful investigation of systematic effects. These aspects are covered in this work and the main achievements are summarised in the following.


The reconstruction of the final state of the \tquark\ pair decay requires the identification of jets originating from \bquark{s}. High space and time resolution in the pixel detector close to the interaction region is vital for these \btag\ techniques. To ensure optimal performance during the currently ongoing \RunTwo, the \RunOne\ pixel detector has been refurbished and the new \gls{IBL} detector has been constructed and inserted as additional innermost layer into the pixel detector. The refurbishment process, the detector commissioning and the performance after reinsertion are described in \chap~\ref{chap:trackingupgrades}.


A well-reconstructed final state lays the basis for the subsequent physics analysis of the \tquark\ mass \mt. Using a performant observable that combines high sensitivity to \mt\ with low susceptibility to systematic uncertainties, a functional parametrisation of \mt\ dependent predictions is established. The template method is used to assess the value of \mt\ and its statistical and systematic uncertainties in the \gls{ATLAS} datasets of the years 2011 and 2012 with \pp\ \cmes\ of $\sqrts=7$ and $8$~\TeV, respectively. The precision is limited by the imperfect knowledge of the jet energy scales.  
%
The analysis at $\sqrts=7$~\TeV\ is presented in \chap~\ref{chap:topmass7TeV}, published in \reference~\cite{Aad:2015nba} and yields a \tquark\ mass value of $\mt = \XZ{173.79}{0.54}{1.30}~\GeV=\XZtot{173.79}{1.41}~\GeV$.
%
The $\sqrts=8$~\TeV\ analysis provides the most precise measurement of \mt\ in the \dil\ channel to date and is presented in \chap~\ref{chap:topmass8TeV}. A phase space optimisation using a multivariate analysis based on the \gls{BDT} technique reduces the total uncertainty on \mt\ to less than $1$~\GeV\ for the first time in \tquark\ mass measurements in the \dil\ channel. The \tquark\ mass is measured to be $\mt = \XZ{\BDTValue}{\BDTStat}{\BDTSyst}~\GeV=\XZtot{\BDTValue}{\BDTTot}~\GeV$, where the central value is blinded.
%
%
Alongside a careful investigation and optimisation of each individual measurement, a combination of measurements yields a substantial improvement in precision by the exploitation of anti-correlations. 
%
% Based on a careful determination of the correlations of systematic effects, the \gls{ATLAS} measurements in the \ljets\ and \dil\ channels are combined to yield a \tquark\ mass value of $\mt = \XZtot{\CombValue}{\CombTot}~\GeV$. %comment for blinded
Based on a careful determination of the correlations of the measurements for all sources of systematic effects, the \gls{ATLAS} measurements in the \ljets\ and \dil\ channels at a \cme\ of $\sqrts=7$~\TeV\ and in the \dil\ channel at $\sqrts=8$~\TeV\ are combined to yield a \tquark\ mass value of $\mt = \XZ{\CombValue}{\CombStat}{\CombSyst}~\GeV = \XZtot{\CombValue}{\CombTot}~\GeV$. 
%
To avoid comparison to existing results while optimising the analysis, the central value of the \dil\ channel analysis at $\sqrts=8$~\TeV\ is blinded by an unknown random but constant shift according to a Gaussian \gls{pdf} with a width of $\Delta\mt=0.5$~\GeV.
%
The details are given in \chap~\ref{chap:combinations}.


With increasing experimental precision, the uncertainties connected with the modelling of \tquark\ production and decay become more and more relevant. However, statistically significant conclusions from a comparison to experimental data are often hindered by the computing-intensive simulation of detector effects. An unfolding technique circumvents the problem by correcting data from \recolevel\ to \stablevel, allowing for a direct comparison of the theory prediction with experimental data. The first steps towards a measurement of the \tquark\ mass at \stablevel\ are performed, setting the basis for a future quantitative comparison of theory predictions in terms of measured \mt\ and a substantial increase in statistical precision of the corresponding systematic uncertainties. This is detailed in \chap~\ref{chap:unfolding}.
%
%
While current \gls{MC} event generators use the \gls{NWA} to factorise \tquark\ production and decay, full \gls{NLO} calculations have become available in recent years. So far, the impact of the new calculations and their connected uncertainties on \tquark\ mass measurements can only be quantified at \genlevel. The presented analysis hints at a possible underestimation of scale variation uncertainties. However, the proof of transferability to \stablevel\ or \recolevel\ requires a successful match to a \gls{PS}, which is still outstanding. This investigation is detailed in \chap~\ref{chap:nlo} and published in \reference~\cite{Heinrich2014}.


To summarise - in this dissertation, the most precise \mt\ measurement to date in the \dileptonic\ \tquark\ pair decay channel has been performed, and measures for an even more precise determination in the future have been presented.
%
%detector upgrades
This includes the pixel detector upgrades to ensure optimal performance in future operation of the \glsname{LHC},
%
%precise determination of systs
the measurement of the \tquark\ mass at \stablevel\ to determine systematic uncertainties with unprecedented statistical precision, and 
%new calculations
the investigation of new calculations with full \gls{NLO} \glsname{QCD} corrections to obtain more precise theory predictions and more reliable uncertainty estimates.
%
%
Together with the large amount of data to be collected in the years to come, this lays the ground for more rigorous constraints on \gls{SM} parameters
%
and thereby for a next step towards the solution of the fundamental problems in the understanding of nature. 

% The years to come will be a decisive period for particle physics. Alongside the exploration of rare phenomena in unexplored energy regimes, a never achieved precision in the determination of the fundamental parameters of the \gls{SM} lies ahead, laying the ground for a next step towards the solution of the fundamental problems in the understanding of nature. 

